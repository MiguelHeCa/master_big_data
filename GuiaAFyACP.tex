\documentclass[11pt,spanish,]{article}
\usepackage{lmodern}
\usepackage{amssymb,amsmath}
\usepackage{ifxetex,ifluatex}
\usepackage{fixltx2e} % provides \textsubscript
\ifnum 0\ifxetex 1\fi\ifluatex 1\fi=0 % if pdftex
  \usepackage[T1]{fontenc}
  \usepackage[utf8]{inputenc}
\else % if luatex or xelatex
  \ifxetex
    \usepackage{mathspec}
  \else
    \usepackage{fontspec}
  \fi
  \defaultfontfeatures{Ligatures=TeX,Scale=MatchLowercase}
    \setmainfont[]{Quicksand-Medium}
\fi
% use upquote if available, for straight quotes in verbatim environments
\IfFileExists{upquote.sty}{\usepackage{upquote}}{}
% use microtype if available
\IfFileExists{microtype.sty}{%
\usepackage{microtype}
\UseMicrotypeSet[protrusion]{basicmath} % disable protrusion for tt fonts
}{}
\usepackage[margin=1in]{geometry}
\usepackage{hyperref}
\hypersetup{unicode=true,
            pdftitle={Guía para AF y ACP},
            pdfauthor={Miguel Hernández},
            pdfborder={0 0 0},
            breaklinks=true}
\urlstyle{same}  % don't use monospace font for urls
\ifnum 0\ifxetex 1\fi\ifluatex 1\fi=0 % if pdftex
  \usepackage[shorthands=off,main=spanish]{babel}
\else
  \usepackage{polyglossia}
  \setmainlanguage[]{spanish}
\fi
\usepackage{color}
\usepackage{fancyvrb}
\newcommand{\VerbBar}{|}
\newcommand{\VERB}{\Verb[commandchars=\\\{\}]}
\DefineVerbatimEnvironment{Highlighting}{Verbatim}{commandchars=\\\{\}}
% Add ',fontsize=\small' for more characters per line
\usepackage{framed}
\definecolor{shadecolor}{RGB}{248,248,248}
\newenvironment{Shaded}{\begin{snugshade}}{\end{snugshade}}
\newcommand{\AlertTok}[1]{\textcolor[rgb]{0.94,0.16,0.16}{#1}}
\newcommand{\AnnotationTok}[1]{\textcolor[rgb]{0.56,0.35,0.01}{\textbf{\textit{#1}}}}
\newcommand{\AttributeTok}[1]{\textcolor[rgb]{0.77,0.63,0.00}{#1}}
\newcommand{\BaseNTok}[1]{\textcolor[rgb]{0.00,0.00,0.81}{#1}}
\newcommand{\BuiltInTok}[1]{#1}
\newcommand{\CharTok}[1]{\textcolor[rgb]{0.31,0.60,0.02}{#1}}
\newcommand{\CommentTok}[1]{\textcolor[rgb]{0.56,0.35,0.01}{\textit{#1}}}
\newcommand{\CommentVarTok}[1]{\textcolor[rgb]{0.56,0.35,0.01}{\textbf{\textit{#1}}}}
\newcommand{\ConstantTok}[1]{\textcolor[rgb]{0.00,0.00,0.00}{#1}}
\newcommand{\ControlFlowTok}[1]{\textcolor[rgb]{0.13,0.29,0.53}{\textbf{#1}}}
\newcommand{\DataTypeTok}[1]{\textcolor[rgb]{0.13,0.29,0.53}{#1}}
\newcommand{\DecValTok}[1]{\textcolor[rgb]{0.00,0.00,0.81}{#1}}
\newcommand{\DocumentationTok}[1]{\textcolor[rgb]{0.56,0.35,0.01}{\textbf{\textit{#1}}}}
\newcommand{\ErrorTok}[1]{\textcolor[rgb]{0.64,0.00,0.00}{\textbf{#1}}}
\newcommand{\ExtensionTok}[1]{#1}
\newcommand{\FloatTok}[1]{\textcolor[rgb]{0.00,0.00,0.81}{#1}}
\newcommand{\FunctionTok}[1]{\textcolor[rgb]{0.00,0.00,0.00}{#1}}
\newcommand{\ImportTok}[1]{#1}
\newcommand{\InformationTok}[1]{\textcolor[rgb]{0.56,0.35,0.01}{\textbf{\textit{#1}}}}
\newcommand{\KeywordTok}[1]{\textcolor[rgb]{0.13,0.29,0.53}{\textbf{#1}}}
\newcommand{\NormalTok}[1]{#1}
\newcommand{\OperatorTok}[1]{\textcolor[rgb]{0.81,0.36,0.00}{\textbf{#1}}}
\newcommand{\OtherTok}[1]{\textcolor[rgb]{0.56,0.35,0.01}{#1}}
\newcommand{\PreprocessorTok}[1]{\textcolor[rgb]{0.56,0.35,0.01}{\textit{#1}}}
\newcommand{\RegionMarkerTok}[1]{#1}
\newcommand{\SpecialCharTok}[1]{\textcolor[rgb]{0.00,0.00,0.00}{#1}}
\newcommand{\SpecialStringTok}[1]{\textcolor[rgb]{0.31,0.60,0.02}{#1}}
\newcommand{\StringTok}[1]{\textcolor[rgb]{0.31,0.60,0.02}{#1}}
\newcommand{\VariableTok}[1]{\textcolor[rgb]{0.00,0.00,0.00}{#1}}
\newcommand{\VerbatimStringTok}[1]{\textcolor[rgb]{0.31,0.60,0.02}{#1}}
\newcommand{\WarningTok}[1]{\textcolor[rgb]{0.56,0.35,0.01}{\textbf{\textit{#1}}}}
\usepackage{graphicx,grffile}
\makeatletter
\def\maxwidth{\ifdim\Gin@nat@width>\linewidth\linewidth\else\Gin@nat@width\fi}
\def\maxheight{\ifdim\Gin@nat@height>\textheight\textheight\else\Gin@nat@height\fi}
\makeatother
% Scale images if necessary, so that they will not overflow the page
% margins by default, and it is still possible to overwrite the defaults
% using explicit options in \includegraphics[width, height, ...]{}
\setkeys{Gin}{width=\maxwidth,height=\maxheight,keepaspectratio}
\IfFileExists{parskip.sty}{%
\usepackage{parskip}
}{% else
\setlength{\parindent}{0pt}
\setlength{\parskip}{6pt plus 2pt minus 1pt}
}
\setlength{\emergencystretch}{3em}  % prevent overfull lines
\providecommand{\tightlist}{%
  \setlength{\itemsep}{0pt}\setlength{\parskip}{0pt}}
\setcounter{secnumdepth}{0}
% Redefines (sub)paragraphs to behave more like sections
\ifx\paragraph\undefined\else
\let\oldparagraph\paragraph
\renewcommand{\paragraph}[1]{\oldparagraph{#1}\mbox{}}
\fi
\ifx\subparagraph\undefined\else
\let\oldsubparagraph\subparagraph
\renewcommand{\subparagraph}[1]{\oldsubparagraph{#1}\mbox{}}
\fi

%%% Use protect on footnotes to avoid problems with footnotes in titles
\let\rmarkdownfootnote\footnote%
\def\footnote{\protect\rmarkdownfootnote}

%%% Change title format to be more compact
\usepackage{titling}

% Create subtitle command for use in maketitle
\providecommand{\subtitle}[1]{
  \posttitle{
    \begin{center}\large#1\end{center}
    }
}

\setlength{\droptitle}{-2em}

  \title{Guía para AF y ACP}
    \pretitle{\vspace{\droptitle}\centering\huge}
  \posttitle{\par}
    \author{Miguel Hernández}
    \preauthor{\centering\large\emph}
  \postauthor{\par}
      \predate{\centering\large\emph}
  \postdate{\par}
    \date{11/8/2019}

\usepackage{booktabs}
\usepackage{longtable}
\usepackage{array}
\usepackage{multirow}
\usepackage{wrapfig}
\usepackage{float}
\usepackage{colortbl}
\usepackage{pdflscape}
\usepackage{tabu}
\usepackage{threeparttable}
\usepackage{threeparttablex}
\usepackage[normalem]{ulem}
\usepackage{makecell}
\usepackage{xcolor}

\begin{document}
\maketitle

\hypertarget{introduccion}{%
\section{Introducción}\label{introduccion}}

Antes de realizar un análisis, lo importante es tener clara la
formulación del problema. En este caso trabajaremos con investigar con
detectar las materias de matemáticas y ciencias naturales pertenecen a
un grupo distinto que materias de francés y latín.

En el cuadro \ref{tab:CL_tab} se muestran las notas de las materias de 8
alumnos, las cuales se presentan de la siguiente manera:

\begin{table}[!h]

\caption{\label{tab:CL_tab}Calificaciones de materias}
\centering
\begin{tabular}{>{\raggedleft\arraybackslash}p{2cm}>{\raggedleft\arraybackslash}p{2cm}>{\raggedleft\arraybackslash}p{2cm}>{\raggedleft\arraybackslash}p{2cm}}
\toprule
Matemáticas & Ciencias naturales & Francés & Latín\\
\midrule
9 & 8 & 6 & 7\\
10 & 9 & 10 & 10\\
3 & 5 & 9 & 8\\
9 & 9 & 8 & 8\\
7 & 6 & 3 & 5\\
\addlinespace
5 & 5 & 5 & 5\\
5 & 5 & 7 & 6\\
4 & 4 & 3 & 4\\
\bottomrule
\end{tabular}
\end{table}

La investigación seguirá la siguiente estructura:

\begin{itemize}
\tightlist
\item
  Análisis de matriz de correlación
\item
  Extracción de factores
\item
  Determinación del número de factores
\item
  Rotación de factores
\item
  Interpretación de factores
\item
  Validación del modelo
\item
  Cálculo de puntuaciones factoriales
\item
  Selección de variables representativas
\item
  Análisis posterior
\end{itemize}

\hypertarget{proceso-en-r}{%
\section{Proceso en R}\label{proceso-en-r}}

\hypertarget{exploracion-de-datos}{%
\subsection{Exploración de datos}\label{exploracion-de-datos}}

Importamos datos que vienen en formato SPSS.

\begin{Shaded}
\begin{Highlighting}[]
\NormalTok{CL =}\StringTok{ }\NormalTok{foreign}\OperatorTok{::}\KeywordTok{read.spss}\NormalTok{(}\StringTok{"data/Ciencias-Letras TOY EJEMPLO.sav"}\NormalTok{, }\DataTypeTok{to.data.frame =}\NormalTok{ T)}
\end{Highlighting}
\end{Shaded}

Calculamos la media y la desviación estándar de cada variable.

\begin{Shaded}
\begin{Highlighting}[]
\CommentTok{# Media}
\KeywordTok{summary}\NormalTok{(CL)}
\end{Highlighting}
\end{Shaded}

\begin{verbatim}
##   Matematicas      Naturales        Francés           Latín       
##  Min.   : 3.00   Min.   :4.000   Min.   : 3.000   Min.   : 4.000  
##  1st Qu.: 4.75   1st Qu.:5.000   1st Qu.: 4.500   1st Qu.: 5.000  
##  Median : 6.00   Median :5.500   Median : 6.500   Median : 6.500  
##  Mean   : 6.50   Mean   :6.375   Mean   : 6.375   Mean   : 6.625  
##  3rd Qu.: 9.00   3rd Qu.:8.250   3rd Qu.: 8.250   3rd Qu.: 8.000  
##  Max.   :10.00   Max.   :9.000   Max.   :10.000   Max.   :10.000
\end{verbatim}

\begin{Shaded}
\begin{Highlighting}[]
\CommentTok{# Desviación estándar}
\KeywordTok{print}\NormalTok{(}\StringTok{"Desviación estándar"}\NormalTok{)}
\end{Highlighting}
\end{Shaded}

\begin{verbatim}
## [1] "Desviación estándar"
\end{verbatim}

\begin{Shaded}
\begin{Highlighting}[]
\KeywordTok{apply}\NormalTok{(CL, }\DecValTok{2}\NormalTok{, sd)}
\end{Highlighting}
\end{Shaded}

\begin{verbatim}
## Matematicas   Naturales     Francés       Latín 
##    2.618615    1.995531    2.615203    1.995531
\end{verbatim}

Matriz de correlaciones

\begin{Shaded}
\begin{Highlighting}[]
\NormalTok{CL_cor =}\StringTok{ }\KeywordTok{cor}\NormalTok{(CL)}
\NormalTok{CL_cor}
\end{Highlighting}
\end{Shaded}

\begin{verbatim}
##             Matematicas Naturales   Francés     Latín
## Matematicas   1.0000000 0.9431723 0.3024774 0.5604357
## Naturales     0.9431723 1.0000000 0.5440581 0.7578475
## Francés       0.3024774 0.5440581 1.0000000 0.9341375
## Latín         0.5604357 0.7578475 0.9341375 1.0000000
\end{verbatim}

\hypertarget{relacion-entre-variables}{%
\subsection{Relación entre variables}\label{relacion-entre-variables}}

La primera aproximación es ver si existen relaciones entre las variables
que esperamos que pertenezcan a sus grupos. En este caso, Matemáticas y
Ciencias naturales están correlacionados; análogamente Francés y Latín.

También se puede ver gráficamente mediante el paquete \texttt{corrplot},
en donde se pone la matriz de correlaciones \texttt{CL\_cor}

\begin{Shaded}
\begin{Highlighting}[]
\NormalTok{corrplot}\OperatorTok{::}\KeywordTok{corrplot}\NormalTok{(CL_cor, }\DataTypeTok{title =} \StringTok{"Matriz de correlaciones de materias"}\NormalTok{)}
\end{Highlighting}
\end{Shaded}

\includegraphics{GuiaAFyACP_files/figure-latex/unnamed-chunk-6-1.pdf}

Otra vía de verlo es ver si en general las variables están relacionadas
mediante el determinante aplicando la función \texttt{det()} a la matriz
de correlaciones \texttt{CL\_cor}. Si el determinante se acerca a 0,
significa que hay relación entre las variables y podemos seguir
explorando si el análisis de factores es adecuado.

\begin{Shaded}
\begin{Highlighting}[]
\KeywordTok{det}\NormalTok{(CL_cor)}
\end{Highlighting}
\end{Shaded}

\begin{verbatim}
## [1] 0.001294676
\end{verbatim}

En efecto, el determinante se acerca a cero y podemos continuar.

\hypertarget{prueba-de-esfericidad}{%
\subsection{Prueba de esfericidad}\label{prueba-de-esfericidad}}

Queremos descartar si los datos tienen una forma esférica. Es decir, una
esfericidad complete se presenta mediante la matriz:

\[
\begin{pmatrix}
  1 & 0 & 0 & 0 \\
  0 & 1 & 0 & 0 \\
  \vdots  & \vdots  & \ddots & \vdots  \\
  0 & 0 & \cdots & 1 
 \end{pmatrix}
\]

Entonces, para descartar que los datos se comportan de esa forma,
utilizamos la prueba de esfericidad de Bartlett, en donde la \(H_0\) es
que los datos presentan esfericidad completa. Es decir, mide el grado en
que la matriz se desvía de la matriz de identidad \(\mathbf{R}\). Para
ello usamos la función \texttt{cortest.bartlett()} del paquete
\texttt{psych}. Los argumentos que requiere esta función son la matriz
de correlaciones \texttt{CL\_cor} y el número de observaciones, que lo
podemos calcular con \texttt{nrow()}.

\begin{Shaded}
\begin{Highlighting}[]
\NormalTok{psych}\OperatorTok{::}\KeywordTok{cortest.bartlett}\NormalTok{(CL_cor, }\KeywordTok{nrow}\NormalTok{(CL))}
\end{Highlighting}
\end{Shaded}

\begin{verbatim}
## $chisq
## [1] 32.13922
## 
## $p.value
## [1] 1.53448e-05
## 
## $df
## [1] 6
\end{verbatim}

Dado que el p-valor es \(< 0.05\), podemos decir que tenemos evidencia
suficiente para rechazar que los datos tienen esfericidad completa y,
por lo tanto, son aptos para el análisis factorial.

También se puede evaluar la esfericidad directamente sobre la matriz
mediante la prueba Kaiser-Mayer-Olkin, la cual mide el cuadrado de los
elementos de la ``imagen'' de la matriz comparada con los cuadrados de
las correlaciones originales.

En nuestro caso, utilizamos la función \texttt{KMO()} del paquete
\texttt{psych} directamente sobre los datos o sobre la matriz de
correlaciones. Esta prueba muestra la Medida de Adecuación del Muestreo
(\texttt{MSA}. por sus siglas en inglés). Kaiser (1974) describió la
interpretación de la \texttt{MSA} de la siguiente manera:

\[
\begin{align*}
\mathrm{KMO} &\geq 0.9 \Rightarrow \mathrm{Estupendo} \\
\mathrm{KMO} &\geq 0.8 \Rightarrow \mathrm{Meritorio} \\
\mathrm{KMO} &\geq 0.7 \Rightarrow \mathrm{Intermedio} \\
\mathrm{KMO} &\geq 0.6 \Rightarrow \mathrm{Mediocre} \\
\mathrm{KMO} &\geq 0.5 \Rightarrow \mathrm{Miserable} \\
\mathrm{KMO} &< 0.5 \Rightarrow \mathrm{Inaceptable}
\end{align*}
\] Aunque otros usuarios la sugieren de esta forma:

\[
\begin{align*}
\mathrm{KMO} &\geq 0.75 \Rightarrow \mathrm{Bien} \\
\mathrm{KMO} &\geq 0.5 \Rightarrow \mathrm{Aceptable} \\
\mathrm{KMO} &< 0.5 \Rightarrow \mathrm{Inaceptable}
\end{align*}
\]

Veamos el resultado.

\begin{Shaded}
\begin{Highlighting}[]
\NormalTok{psych}\OperatorTok{::}\KeywordTok{KMO}\NormalTok{(CL)}
\end{Highlighting}
\end{Shaded}

\begin{verbatim}
## Kaiser-Meyer-Olkin factor adequacy
## Call: psych::KMO(r = CL)
## Overall MSA =  0.62
## MSA for each item = 
## Matematicas   Naturales     Francés       Latín 
##        0.57        0.65        0.57        0.66
\end{verbatim}

Podemos observar que la MSA es mediocre para Kaiser, y aceptable para la
mayoría de los usuarios. Digamos que mientras no sea inaceptable,
podemos seguir con el análisis.

\begin{center}\rule{0.5\linewidth}{\linethickness}\end{center}

\hypertarget{refs}{}
\leavevmode\hypertarget{ref-Kaiser1974}{}%
Kaiser, Henry F. 1974. «An index of factorial simplicity».
\emph{Psychometrika} 39 (1): 31-36.
\url{https://doi.org/10.1007/BF02291575}.


\end{document}
